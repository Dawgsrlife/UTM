\documentclass{article}
\usepackage{amsthm,amsmath,amsfonts,lipsum}
\usepackage[T1]{fontenc}
\usepackage{beramono}
\usepackage{listings}
\usepackage{fontawesome5}
\usepackage{adjustbox}
\usepackage{mathabx}
\usepackage{thmtools}
\usepackage{import}
\usepackage{graphicx}
\usepackage{setspace}
\usepackage{geometry}
\usepackage{physics}
\usepackage{float}
\usepackage[english]{babel}
\usepackage{framed}
\usepackage[dvipsnames]{xcolor}
\usepackage{tcolorbox}
\usepackage{fancyhdr}
\usepackage{hyperref}
\usepackage{booktabs}
\usepackage{enumitem}

% Define custom tcolorbox styles for notes
\tcbuselibrary{skins, breakable}
\newtcolorbox{definitionbox}{colframe=RoyalBlue, colback=blue!5!white, title=Definition}
\newtcolorbox{examplebox}{colframe=ForestGreen, colback=green!5!white, title=Example}
\newtcolorbox{notebox}{colframe=RedOrange, colback=orange!5!white, title=Note}
\newtcolorbox{theorembox}{colframe=RoyalPurple, colback=purple!5!white, title=Theorem}

% Geometry settings
\geometry{letterpaper, portrait, includeheadfoot=true, hmargin=1in, vmargin=1in}
\onehalfspacing

% Header and footer
\pagestyle{fancy}
\fancyhf{}
\lhead{MAT232 - Lecture Notes}
\rhead{\thepage}
\lfoot{University of Toronto Mississauga}
\rfoot{\today}

% Document starts
\begin{document}
\renewcommand{\familydefault}{\rmdefault}
%===================================================================================
% TITLE PAGE
% Edit the placeholders below for each problem set
%===================================================================================

\begin{titlepage}
    \null
    \vfill
    \begin{center}

        % Course Code - Edit this for each course
        {\fontsize{40}{48}\selectfont \bfseries [COURSE CODE]}
        \vspace{10pt} \\

        % Problem Set Number - Edit this for each assignment
        {\LARGE Problem Set [NUMBER]} \\
        \vspace{15pt}

        % Student Information - Edit once per semester
        {\large \textbf{AlexanderTheMango}} \\
        \vspace{5pt}
        {\normalsize Student ID: [YOUR STUDENT ID]} \\
        \vspace{10pt}

        % University Information
        {\normalsize University of Toronto Mississauga} \\
        \vspace{5pt}

        % Due Date - Edit for each assignment
        {\normalsize Due: [DUE DATE]}

    \end{center}
    \vfill
\end{titlepage}

% Start content on new page
\newpage
\pagebreak

\normalsize

\section*{Lecture Title}
\begin{notebox}
This template is designed for MAT232 lecture notes. Replace this content with your specific lecture details.
\end{notebox}

\section*{Key Concepts}
\begin{definitionbox}
A \textbf{parametric equation} is a set of equations that express the coordinates of the points of a curve as functions of a variable, called a parameter.
\end{definitionbox}

\section*{Examples}
\begin{examplebox}
\textbf{Example 1:} Consider the parametric equations:
\[ x = t, \quad y = t^2, \quad t \in \mathbb{R}. \]
\begin{itemize}
    \item At $t = 0$, $(x, y) = (0, 0)$.
    \item At $t = 1$, $(x, y) = (1, 1)$.
\end{itemize}
This describes a parabola.
\end{examplebox}

\section*{Theorems and Proofs}
\begin{theorembox}
\textbf{Theorem:} If $x(t)$ and $y(t)$ are differentiable functions, the slope of the curve is given by:
\[ \frac{dy}{dx} = \frac{\frac{dy}{dt}}{\frac{dx}{dt}}, \quad \text{provided } \frac{dx}{dt} \neq 0. \]
\end{theorembox}

\section*{Additional Notes}
\begin{notebox}
Always check the domain of the parameter $t$ when solving problems involving parametric equations.
\end{notebox}

\end{document}
